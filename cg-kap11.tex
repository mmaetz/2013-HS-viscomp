\chapter{Shadows}
\section{Why important}
\begin{compactdesc}
	\item[\lp{depth cue}] 
	\item[\lp{Scene lighting}] Light position, Point vs. Area light.
	\item[\lp{Realism}]
\section{Basic shadows}
	\item[\lp{planar shadows}] draw projection of the object on the ground.\\
		Limitations: \\
		\begin{enumerate*}[label=\protect\circled{\arabic*},itemjoin=]
			\item self shadows\\
			\item shadows on other objects\\
			\item curved surfaces
		\end{enumerate*}
	\item[\lp{projective tetxure shadows}] Separate obstacle and receiver. Compute b/w  image of the obstacle from light. Use image as projective texture. Limitations: Need to specify obstactle \& receiver, no self-shadows.
\section{Shadow maps}
In high-end production software and games. \\
\begin{enumerate*}[label=\protect\circled{\arabic*},itemjoin=]
	\item Compute the depths from the light\\
	\item Compute the depths from the camera
\end{enumerate*}
\item[\lp{Algorithm}]
For each pixel on the camera plane\\
\begin{enumerate*}[label=\protect\circled{\arabic*},itemjoin=]
	\item Comptue the point in world coordinates\\
	\item Project point onto the light plane\\
	\item Compare $d(\vtr{x}_L)$ (shadow map) and $z_L$\\
	\item If $d(\vtr{x}_L)<z_L$, $\vtr{x}$ is in shadow
\end{enumerate*}
\pgr{5-1_VisibilityAndShadows}{17}{0.65}{-1.32}{3.0}{0.45}
Depth map rendered from the light
\pgr{5-1_VisibilityAndShadows}{18}{-2.5}{-1.32}{0.2}{0.27}
Rendering from the camera
\pgr{5-1_VisibilityAndShadows}{18}{-0.2}{-1.32}{2.5}{0.27}
	\item[\lp{Limitations}]\hfill \\
		\emph{Bias}:\\
		For a visible point $d(\vtr{x}_L)<z_L$, how to avoid self-shadowing? Add bias
		\begin{gather*}
			d(\vtr{x}_L)+\text{bias}<z_L.
		\end{gather*}
		Choosing a good bias can be very tricky.
		\emph{Field of view:} \\A point to shadow can be outside the field of view of the shadow map. $\to$ Use cubical shadow map or spot lights.\\
		\emph{Aliasing:}\\
		Undersampling of the shadow map
	\item[\lp{Filtering}] Should we filter depth? No. Instead, filter the result of the test
		\begin{gather*}
			d(\vtr{x}_L)+\text{bias}<z_L
		\end{gather*}
		Take a weighted average of comparisons. Bigger filter produces fake soft shadown. Setting bias is tricky.
\section{Shadow volumes}
Explicitly represent the volume of space in shadow. If a polygon is inside the volume, it is in shadow. Similar to clipping. Naive implementation $\mathcal{O}(\#polygons\times\#lights)$.
\item[\lp{Algorithm}]\hfill\\
	\begin{enumerate*}[label=\protect\circled{\arabic*},itemjoin=]
		\item Shoot a ray from the eye\\
		\item Incre-/decrement a counter each time boundary of shadow volume is intersected\\
		\item If counter$>0$, primitive is in shadow.
		\item If counter$=0$, primitive is not in shadow
	\end{enumerate*}
	Optimization: Use silhouette edges only (where a back-facing \& front-facing polygon meet).
\item[\lp{Limitations}]\hfill\\
	\begin{enumerate*}[label=\protect\circled{\arabic*},itemjoin=]
		\item Introduces a lot of new geometry\\
		\item Expensive to rasterize long skinny triangles\\
		\item Objects must be watertight to use the silhouette optimization\\
		\item Rasterization of polygons sharing an edge must not overlap and not have a gap
	\end{enumerate*}
	\pgrr{5-1_VisibilityAndShadows}{30}{-3}{-1.6}{3}{1.5}{90}
\end{compactdesc}
