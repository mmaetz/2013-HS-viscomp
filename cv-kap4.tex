\chapter{Image features}
\section{Template matching}
\begin{compactdesc}
	\item[\lp{Problem}] Locate an object, described by a template $t(x,y)$, in the image $s(x,y)$. \emph{Example:} Passport photo as image and eyes to detect.
	\item[\lp{Method}] Search for the best match by minimizing mean -squared error $E(p,q)$
		\begin{align*}\scriptscriptstyle
			&={\scriptstyle\sum\limits_{\mathclap{x,y=-\infty}}^{\infty}}\left[ s(x,y)\!-\!t(x\!-\!p,y\!-\!q) \right]^2\\
			&={\scriptstyle\sum\limits_{\mathclap{x,y=-\infty}}^{\infty}}\abs{s(x,y)}^2+\abs{t(x,y)}^2\\
			&\,-2\!\cdot\!{\scriptstyle\sum\limits_{\mathclap{x,y=-\infty}}^{\infty}}s(x,y)\cdot t(x\!-\!p,y\!-\!q)
		\end{align*}
		Equivalently, maximize \emph{area correlation}
		\begin{align*}
			r(p,q)&={\scriptstyle\sum_{x,y=-\infty}^{\infty}}s(x,y)\cdot t(x-p,y-q)\\
			&=s(p,q)*t(-p,-q)\\
			\!\!\!\!\leq&\sqrt{\!\!\left[\! {\scriptstyle\sum} \abs{s(x,y)}^2 \!\right]\!\!\cdot\!\!\left[\! {\scriptstyle\sum} \abs{t(x,y)}^2 \!\right]\!\!},
		\end{align*}
		where in the last step the Cauchy-Schwarz inequality was used. Equality $\Leftrightarrow$
		\begin{gather*}
			s(x,y)=\alpha\cdot t(x-p,y-q)\quad \text{with }\alpha\geq0
		\end{gather*}
		Area correlation is equivalent to convolution of image $s(x,y)$ with impulse response $t(-x,-y)$. 
	\item[\lp{Diagram of template matcher}] $\xrightarrow{s(x,y)}t(-x,-y)\xrightarrow{r(x,y)}\displaystyle
		\begin{matrix}
			\text{search}\\
			\text{peak(s)}
		\end{matrix}
		\to
		\begin{matrix}
			\text{object}\\
			\text{location(s) }p,q
		\end{matrix}
		$ Remove mean before template matching to avoid bias towards bright image areas.
\end{compactdesc}
\section{Edge detection}
	 Idea (continuous-space): Detect local gradient
		\begin{gather*}
			\norm{\nabla(f(x,y))}=\sqrt{\left( \partial_{x}f \right)^2+\left( \partial_{y}f \right)^2}
		\end{gather*}
		Digital image: Use finite differences instead:\\
		\begin{inparadesc}
			\item[difference] 
				$\left(\begin{smallmatrix}
					-1&1
				\end{smallmatrix}\right)$,
			\item[central difference] 
				$\left(\begin{smallmatrix}
					-1&[0]&1
				\end{smallmatrix}\right)$;
			\item[Prewitt] 
				$\left(\begin{smallmatrix}
					-1&0&1\\
					-1&[0]&1\\
					-1&0&1
				\end{smallmatrix}\right)$,
				$\left(\begin{smallmatrix}
					-1&-1&1\\
					0&[0]&0\\
					-1&-1&1
				\end{smallmatrix}\right)$;
			\item[Sobel]
				$\left(\begin{smallmatrix}
					-1&0&1\\
					-2&[0]&2\\
					-1&0&1
				\end{smallmatrix}\right)$,
				$\left(\begin{smallmatrix}
					-1&-2&1\\
					0&[0]&0\\
					-1&-2&1
				\end{smallmatrix}\right)$;
			\item[Roberts]
				$\left(\begin{smallmatrix}
					[0]&1\\
					-1&0\\
				\end{smallmatrix}\right)$,
				$\left(\begin{smallmatrix}
					[1]&0\\
					0&-1\\
				\end{smallmatrix}\right)$,
		\end{inparadesc}
\begin{compactdesc}
	\item[\lp{Laplacian operator}] Detects discontinuities by considering second derivative
		\begin{gather*}
			\nabla^2 f(x,y)=\partial_{x}^{2}f(x,y)+\partial_{y}^{2}f(x,y).
		\end{gather*}
		Isotropic (rotationally invariant) operator, zero-crossings mark edge location, discrete-space approximation by convolution with $3\times 3$ impulse response
		$\left(\begin{smallmatrix}
			0&1&0\\
			1&[-4]&1\\
			0&1&0\\
		\end{smallmatrix}\right)$, or
		$\left(\begin{smallmatrix}
			0&1&0\\
			1&[-8]&1\\
			0&1&0\\
		\end{smallmatrix}\right)$,
	\item[\lp{Laplacian of Gaussian}]
		The Laplacian operator is very sensity to fine detail and noise, so blur it first with Gaussian.$\to$ do it in one operator Laplacian of Gaussian (LoG)
		\begin{align*}
			&\LoG(x,y)\\
			&\quad=-\frac{1}{\pi\sigma^4}\left[ 1-\frac{x^2+y^2}{2\sigma^2} \right]\\
			&\qquad \cdot e^{-\frac{x^2+y^2}{2\sigma^2}}
		\end{align*}
\end{compactdesc}
\subsection{Canny edge detector}
\begin{inparaenum}[\itshape(1)]
	\item Smooth image with a gaussian filter
	\item Compute gradient magnitude and angle (Sobel, Prewitt,\ldots)
		\begin{align*}
			M(x,y)&=\sqrt{\left( \partial_xf \right)^2+\left( \partial_yf \right)^2},\\
			\alpha(x,y)&=\arctan\left( \partial_yf/\partial_xf \right)
		\end{align*}
	\item Apply nonmaxima suppression to gradient magnitude image
	\item Double tresholding to detect strong and weak edge pixels
	\item Reject weak edge pixels not connected with strong edge pixels
\end{inparaenum}
\begin{compactdesc}
	\item[\lp{Canny nonmaxima suppression}] Quantize edge normal to one of four directions: horizontal, $\SI{-45}{\degree}$, vertical, $\SI{+45}{\degree}$. If $M(x,y)$ is smaller than either of its neighbors in edge normal direction $\to$ suppress; else keep
	\item[\lp{Double-thresh. of grad. magn.}]\mbox{}
		\begin{compactdesc}
			\item[strong edge:] $M(x,y)\geq\theta_{\text{high}}$
			\item[weak edge:] $\theta_{\text{high}}>M(x,y)\geq\theta_{\text{low}}$
		\end{compactdesc}
		Typical setting: $\theta_{\text{high}},\theta_{\text{low}}=2,3$. Region labeling of edge pixels. Reject regions without strong edge pixels.
\end{compactdesc}
\section{Feature detection}
\subsection{Hough transform}
\emph{Problem:} fit a straight line (or curve) to a set of edge pixels. Hough transform (1962): generalized template matching technique. 
\begin{inparaenum}[\itshape(1)]
	\item Consider detection of straight lines $y=mx+c$.
	\item draw a line in the parameter space $m,c$ for each edge pixel $x,y$ and increment bin counts along line. Detect peak(s) in $(m,c)$ plane.
	\item Alternative parametrization avoids infinite-slope problem $x\cos\theta+y\sin\theta=\rho$
\end{inparaenum}
\begin{compactdesc}
	\item[\lp{circle detection}] find circles of fixed radius $r$. For circles of undetermined radius, use 3d Hough transform for parameters $(x_0,y_0,r)$
\end{compactdesc}
\subsection{Detecting corner points}
Many applications benefit from features localized in $(x,y)$. Edges well localized only in one direction $\to$ detect corners. Desirable properties of corner detector:
\begin{inparaenum}[\itshape(1)]
	\item Accurate localization,
	\item	invariance against shift, rotation, scale, brightness change, 
	\item robust against noise, high repeatability
\end{inparaenum}
\subsection{Most accurately localizable patterns}
\begin{compactdesc}
\item[\lp{Local displacement sensitivity}] 
	\begin{gather*}
		S(\Delta x,\Delta y) ={\scriptstyle\sum\limits_{\mathclap{x,y\in \text{window}}}}^{}\left[ f(x,y)-f(x-\Delta x,y+\Delta y) \right]^2
	\end{gather*}
\item[\lp{Linear approximation for small $\Delta x,\Delta y$}]
	\begin{multline*}
		f(x+\Delta x,y+\Delta x)\\\approx f(x,y)+\partial_x f(x,y)\Delta x+\partial_y(x,y)\Delta y
	\end{multline*}
	\begin{align*}
		S(\Delta x,\Delta y)&\approx{\scriptstyle\sum\limits_{\mathclap{(x,y)\in\text{window}}}}^{}
		\left[
		\begin{pmatrix}
			\partial_x f&\partial_y f
		\end{pmatrix}
		\begin{pmatrix}
			\Delta x\\
			\Delta y
		\end{pmatrix}
		\right]\\
		&=
		\begin{pmatrix}
			\Delta x&\Delta y
		\end{pmatrix}
		\vtr{M}
		\begin{pmatrix}
			\Delta x\\
			\Delta y
		\end{pmatrix}
	\end{align*}
	\item[\lp{Feature point extraction}]
		\begin{align*}
			\text{SSD}&\approx\Delta^{T}M\Delta\\
			\shortintertext{Find points for which the following is large}
			\min \Delta^{T}M\Delta
		\end{align*}
		for $\norm{\Delta}=1$. i.e. maximize eigenvalues of $M$.
	\item[\lp{Keypoint detection}]
		Often based on eigenvalues $\lambda_1,\lambda_2$ of $M$ (``structure matrix''/``normal matrix''/second-moment matrix'')
		\begin{gather*}
			M={\scriptstyle\sum\limits_{\mathclap{(x,y)\in \text{window}\quad\,\,}}}^{}
			\begin{pmatrix}
				\left( \partial_x f \right)^2&\partial_xf\partial_yf\\
				\partial_xf\partial_yf&\left( \partial_y f \right)^2
			\end{pmatrix},
		\end{gather*}
		Measure of ``cornerness''
		\begin{align*}
			C(x,y)&=\det(M)-k\cdot \left( \trace\, M \right)^2\\
			&=\lambda_1\lambda_2-k\cdot(\lambda_1+\lambda_2)
		\end{align*}
		\pgr{VisComp03b_Features}{41}{0.37}{-1.67}{3.0}{1.20}
	\item[\lp{Corner importance weight}] Give more importance to central pixels by using Gaussian weighting function
		\begin{align*}
			M&={\scriptstyle\sum\limits_{\mathclap{x,y\in\text{window}}}}^{}G(x-x_0,y-y_0,\sigma)\\
			&\quad\cdot
			\begin{pmatrix}
				\left( \partial_xf \right)^2&\partial_xf\partial_yf\\
				\partial_xf\partial_yf&\left( \partial_yf \right)^2
			\end{pmatrix}
		\end{align*}
		Compute subpixel localization by fitting parabola to \emph{cornerness function}
	\item[\lp{Robustness of Harris corner detector}]
		\begin{inparaenum}[\itshape(1)]
			\item Invariant to brightness offset: $f(x,y)\to f(x,y)+c$
			\item Invariant to shift and rotation
			\item Not invariant to scaling
		\end{inparaenum}
		\subsection{Lowe's SIFT features}
		Recover features with position, orientation and scale. 
	\item[\lp{Position}] 
		\begin{inparaenum}[\itshape(1)]
		\item Look for strong responses of DoG filter,
		\item only consider local maxima.
		\end{inparaenum}
	\item[\lp{Scale}] 
		\begin{inparaenum}[\itshape(1)]
			\item Look for strong responses of DoG filter over scale space. 
			\item only consider local maxima in both position and scale. 
			\item Fit quadratic around maxima for subpixel accuracy.
		\end{inparaenum}
	\item[\lp{Orientation}] 
		\begin{inparaenum}[\itshape(1)]
			\item Create histogram of local gradient directions computed at selected scale.
			\item Assign canonical orientation at peak of smoothed histogram.
			\item Each key specifies stable 2D coordinates ($x$,$y$,scale,orientation)
		\end{inparaenum}
	\item[\lp{SIFT descriptior}]
		\begin{inparaenum}[\itshape(1)]
			\item Thresholded image gradients are sampled over $16\times16$ array of locations in scale space.
			\item Create array of orientation histograms
			\item 8 orientations $\times$ $4\times4\text{ histogram array}=128$ dimensions
		\end{inparaenum}
\end{compactdesc}
