\chapter{Clipping}
Removing parts of objects outside the viewing frustum.
\pgr{5-1_Clipping}{2}{-0.2}{-1.3}{2.8}{0.2}
Why clip?\\
\begin{enumerate*}[label=\protect\circled{\arabic*},itemjoin=]
	\item Save computation\\
	\item Memory consistency\\
	\item Stability: Division by 0
\end{enumerate*}
\begin{compactdesc}
	\item[\lp{Culling}] Rejects via bounding volumes.
	\item[\lp{Clipping}] Partially visible objects need to be clipped.
\section{Parametric clipping}
Parametric clipping (Liang-Barsky Algorithm). Idea: Compute intersection in 1D parameter space of the line
	\begin{gather*}
		\vtr{p}(t)=\vtr{p}_0+(\vtr{p}_1-\vtr{p}_0)t,\qquad t\in[0,1]
	\end{gather*}
	Outside of clip region
	\begin{gather*}
		\vtr{n}_{i}^{T}\!\left( \vtr{p}(t)-\vtr{p}_{E_i} \right)>0.
	\end{gather*}
\pgr{5-1_Clipping}{4}{-0.7}{-1.3}{3}{0.15}
Intersection point
\begin{gather*}
	\vtr{n}_{i}^{T}\!\left(\vtr{p}(t)-\vtr{p}_{E_i}  \right)=0.
\end{gather*}
Insert
\begin{align*}
	t&=\frac{\vtr{n}_{i}^{T}(\vtr{p}_0-\vtr{p}_{E_i})}{-\vtr{n}_{i}^{T}\vtr{d}},\\
	\vtr{d}&=\vtr{p}_1-\vtr{p}_0.
\end{align*}
Classification of entry points (PE) and leaving points (PL)
		\pgr{5-1_Clipping}{6}{-2.3}{-1.3}{2.4}{0.4}
	\item[\lp{Algorithm}]\hfill\\
		\begin{enumerate*}[label=\protect\circled{\arabic*},itemjoin=]
			\item Test for all $t\in[0,1]$\\
			\item Entry and leaving points computed by 
				$\centering
				\begin{aligned}
					\vtr{n}^{T}&<0 \Rightarrow \text{PE}\\
					\vtr{n}^{T}\vtr{d}&>0\Rightarrow\text{PL}
				\end{aligned}
				$
			\item Compute maximum and minimum
				$
				\begin{aligned}
					t_{E}&=\max(t_{E_i},\vtr{p}_{E_i}),\\
					t_{L}&=\min(t_{L_i},\vtr{p}_{L_i}).
				\end{aligned}
				$\\
			\item If $t_{E}>t_L \Rightarrow$ no relevant intersection. 
		\end{enumerate*}
		\pgrr{5-1_Clipping}{8}{-1.5}{-1.3}{1.3}{0.4}{90}
		\begin{lstlisting}
void Clip2D (float *x0, float *y0, float *x1, float *y1, boolean *visible)
{
	float tE, tL;
	float dx, dy;
	dx = *x1 - *x0;
	dy = *y1 - *y0;
	*visible = F\leftarrowE;
	if (dx &=&  0 && dy &=&  0 && ClipPoint(*x0, *y0))
	*visible = TRUE;
	else {
		tE = 0;
		tL = 1;
		if (CLIPt(dx, xmin - *x0, &tE, &tL))
		if (CLIPt(-dx, *x0 - xmax, &tE, &tL))
		if (CLIPt(dy, ymin - *y0, &tE, &tL))
		if (CLIPt(-dy, *y0-ymax, &tE, &tL)) {
			*visible = TRUE;
			if (tL < 1) {
				*x1 = *x0 + tL * dx;
				*y1 = *y0 + tL * dy;
			}
			if (tE > 0) {
				*x0 = *x0 + tE * dx;
				*y0 = *y0 + tE * dy;
			}
		}
	}
}
		\end{lstlisting}
		\begin{lstlisting}
boolean CLIPt (float denom, float num, float *tE, float *tL) {
float t;
boolean accept;
accept = TRUE;
if (denom > 0) {
/* entry */
t = num/denom;
if (t > *tL)
accept = FALSE;
else if (t > *tE)
*tE = t;
}
else if (denom < 0) {
/* leave */
t = num/denom;
if (t < *tE)
accept = FALSE;
else if (t < *tL)
*tL = t;
}
else
/* parallel */
if (num > 0)
accept = FALSE;
return accept;
}
		\end{lstlisting}
	\item[\lp{Polygon clipping}] Extension of Liang-Barsky line clipping algorithm: Liang-Barsky polygon clipping algorithm\\
		\begin{enumerate*}[label=\protect\circled{\arabic*},itemjoin=]
			\item Walk around polygon\\
			\item For each edge, output endponts of visible part\\
			\item Sometimes turning vertex is needed!
		\end{enumerate*}
\pgr{5-1_Clipping}{11}{0.2}{-1.3}{3}{0.7}
		Uses subdivision of the window plane. Turning vertex needed when polygon touches an ``inside 2'' region. Unnecessary vertices do not cause artifacts inside visible window. Segment in parametric form
		\begin{gather*}
			\vtr{p}(t)=\vtr{p}_0+(\vtr{p}_1-\vtr{p}_0)t
		\end{gather*}
		In general, corresponding line cuts 4 times.
\pgr{5-1_Clipping}{15}{-2.7}{-1.45}{3}{0.1}
(Talking about left image above with upper right corner colored blue.) Part of segment visible
\begin{gather*}
	t_{\text{out}1}>0\quad t_{\text{in}2}\leq 2,
\end{gather*}
Enter ``inside 2'' region
\begin{gather*}
	0<t_{\text{out}2}\leq 1.
\end{gather*}
(Talking about right image above with upper and bottom right and corner colored blue.)
Segment not visible. Enter ``inside 2'' region
\begin{gather*}
	0<t_{\text{out}1}\leq 1,\qquad 0<t_{\text{out}2}\leq 1.
\end{gather*}
Liang-Barsky polygon clipping algorithm. C-Code:
\begin{lstlisting}
for each edge e {
compute tin1, tin2, tout1, tout2;
if (tin2 > tout1) { /* kein sichtbares Segment */
if (0 < tout1 <= 1) Output_vert(turning vertex);
}
else {
if ((0 < tout1) && (1 >= tin2)) { /* sichtbares Segment vorhanden */
if (0 <= tin2)
Output_vert(appropriate side intersection);
else
Output_vert(starting vertex);
if (1 >= tout1)
Output_vert(appropriate side intersection);
else
Output_vert(ending vertex);
}
}
if (0 < tout2 <= 1) Output_vert(appropriate corner);
}
\end{lstlisting}
\item[\lp{3D clipping}] Clipping algorithms work for rectangular windows. But: View frustum is not rectangular. Solution: CLip after perspective division in homogeneous coordinates.
	For $\xi\in\left\{ x,y,z \right\}$
	\begin{gather*}
		-1\leq \frac{\xi}{w}\leq 1.
	\end{gather*}
\end{compactdesc}
