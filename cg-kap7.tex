\chapter{Texture mapping}
Enhance details without increasing geometric complexity
\begin{gather*}
	\text{texture}+\text{geometry}=\text{rendering}
\end{gather*}
\begin{compactdesc}
	\item[\lp{Issues}]\hfill\\
\begin{enumerate*}[label=\protect\circled{\arabic*},itemjoin=]
	\item Mapping between texture and geometry\\
	\item Anti-aliasing and filtering\\
	\item Level-of detail\\
	\item One-to-one mapping between texture and geometry\\
\end{enumerate*}
\section{Parametrization}
	\item[\lp{sphere}] 
		\begin{gather*}
			\!\left(\!\begin{smallmatrix}
				u\\
				v
			\end{smallmatrix}\!\right)\!
			\to
			\!\left(\!\begin{smallmatrix}
				\sin u\sin v\\
				\cos v\\
				\cos u \sin v
			\end{smallmatrix}\!\right)\!
		\end{gather*}
	\item[\lp{Desirable properties}]\hfill\\
		\begin{enumerate*}[label=\protect\circled{\arabic*},itemjoin=]
			\item Low distortion\\
			\item Bijective mapping\\
			\item Efficient to compute
		\end{enumerate*}
	\item[\lp{Issues}] Finding cuts, texture atlases.
	\item[\lp{OBJ files}]\hfill\\
		\begin{lstlisting}
OBJ Files
v 0.131171 -0.113469 0.178314
v 0.130945 -0.114951 0.182474
...
vt 0.538446 0.4275
vt 0.550132 0.41427
...
vn 0.609697 0.486474 0.625789
vn 0.799934 0.334347 0.498315
...
f 22/209/22 220/210/220
221/211/221
f 21/213/21 219/214/219
220/210/229
...
		\end{lstlisting}
		With ``v'' the vertex positions, ``vt'' the texture coordinates, ``f'' the faces (triagles) coor\#/tex\#/normal\#.
	\item[\lp{OpenGL}]\hfill\\
		Load and bind texture
		\begin{lstlisting}
loadImage(&texture_data);
glGenTextures(1, &texId);
glBindTexture(GL_TEXTURE_2D, texId);
glTexImage2D(GL_TEXTURE_2D, 0, GL_RGB, w, h, 0, GL_RGB, GL_UNSIGNED_BYTE, texture_data);
...
		\end{lstlisting}
		Fragment shader: texture lookup
		\begin{lstlisting}
uniform sampler2D texWood;
varying vec2 texCoords; // [u, v]
void main(void) {
gl_FragColor = texture2D(texWood,
texCoords);
}	
		\end{lstlisting}
		\section{Light maps}
	\pgr{4-2_GeometryAndTexturesII}{12}{-2}{-1.1}{2}{-0.2}
		Introduced in Quake. Simulates the effect of a local light source. Can be pre-computed and dynamically adapted. 
		\section{Environment map}
		Rendering reflective objects efficiently. Texture coordinates are computed by intersecting the reflected ray with the surrounding sphere.
	\pgr{4-2_GeometryAndTexturesII}{14}{-2}{-1.3}{2}{0.42}
	Utilizing a cube allows simpler computations. Photos can be used as environment maps.
\section{Texture Filtering}
	\item[\lp{aliasing in computer graphics}]\hfill\\
		\begin{enumerate*}[label=\protect\circled{\arabic*},itemjoin=]
			\item jaggy edges\\
			\item moiré patterns\\
			\item loss of detail
		\end{enumerate*}
		Low-pass filtering avoids aliasing. Typical low-pass filters
		\begin{gather*}
			f(x,y)=\frac{1}{2\pi\sigma^2}e^{-\left( \frac{x^2+y^2}{2\sigma^2} \right)}
		\end{gather*}
	\item[\lp{Filtering in texture vs. image space}] Use anisotropic spatialy-varying texture filtering. Like the anisotropic 2D Gaussian
		\begin{gather*}
			f(x,y)=\frac{1}{2\pi \sigma_x\sigma_y}e^{-\frac{x^2}{2\sigma_{x}^{2}}-\frac{y^2}{2\sigma_{y}^{2}}}
		\end{gather*}
\section{Bump mapping}
Perturb surface normal according to textures. Represents small-scale geometry.
	\pgr{4-2_GeometryAndTexturesII}{18}{-3}{-1.2}{0}{0.2}
	\begin{align*}
		\vtr{n}&=\pdv{\vtr{p}}{u}\times\pdv{\vtr{p}}{v}=\vtr{p}_{u}\times\vtr{p}_{v},\\
		\vtr{p}'\!&=\vtr{p}+\frac{b\vtr{n}}{\abs{\vtr{n}}},\\
		\vtr{n}'\!&=\pdv{\vtr{p}'}{u}\times\pdv{\vtr{p}'}{v},\\
		\pdv{\vtr{p}'}{u}&=\pdv{\vtr{p}}{u}+\pdv{}{u}\frac{(b\vtr{n})}{\abs{\vtr{n}}},\\
		\vtr{n}'\!&=\vtr{n}+b_u(\vtr{n}\times\vtr{p}_u)\\
		&\quad+b_v(\vtr{n}\times\vtr{p}_v).
	\end{align*}
	\pgr{4-2_GeometryAndTexturesII}{20}{-3}{-1.2}{-0.9}{0.4}
	\begin{gather*}
		b_{ij}\coloneqq b(u_i,v_j),
	\end{gather*}
	\begin{align*}
		b_u&=\frac{b_{21}-b_{11}+b_{22}-b_{12}}{2\Delta u},\\
		b_v&=\frac{b_{12}-b_{11}+b_{22}-b_{21}}{2\Delta v}.
	\end{align*}
\item[\lp{Limitations}] \hfill\\
	\begin{enumerate*}[label=\protect\circled{\arabic*},itemjoin=]
		\item No bumps on the silhouette\\
		\item No self-occlusions\\
		\item No self-shadowing
	\end{enumerate*}
\section{Displacement mapping}
Perturb geometry based on textures.
\section{Procedural textures}
	\item[\lp{Perlin Noise}] Mixes different levels of noise.
		\pgr{4-2_GeometryAndTexturesII}{25}{-2.7}{-1.3}{2.7}{0.7}
	\item[\lp{Gabor noise}] More control over the spectral properties.
		\pgr{4-2_GeometryAndTexturesII}{27}{-3}{-1.2}{0}{0.5}
	\item[\lp{3D textures}](They exist.)\\
\section{Mip-Mapping}
Store down-sampled versions of a texture. Use low-res versios for far away objects. Interpolate for in-between depths. Avoids aliasing and improves efficiency. Compute lower resolution versions by weighted averages.
\section{Perspective interpolation}
From texture coordinates of \emph{vertices} to texture coordinates of \emph{pixels}. Linear interpolation in screen-space. 
	\pgr{4-2_GeometryAndTexturesII}{32}{-2}{-1.3}{2}{-0.2}
	Linear variation in world coordinates yields non-linear variation in screen coordinates.
	\pgr{4-2_GeometryAndTexturesII}{33}{-2}{-1.35}{2}{0.3}
\end{compactdesc}
