\chapter{Image filtering}
Image filtering is modifying the pixels in an image based on some function of a local neighborhood of the pixels.
\section{Linear Shift-Invariant Filtering}
About modifying pixels based on \emph{neighborhood}. Local methods simplest. Linear means \emph{linear combination} of neighbors. Linear methods simplest. \emph{Shift-invariant} means doing the same for each pixel. Same for all is simplest. Useful to: Low-level image processing operations, smoothing and noise reduction, sharpen, detect or enhance features.
\begin{compactdesc}
	\item[\lp{Linear operation}] $L$ is a \emph{linear} operation if 
		\begin{gather*}
			L\!\left[ \alpha I_1+\beta I_2 \right]=\alpha L[I_1]+\beta L[I_2]
		\end{gather*}
	\item[\lp{Output}] $I'$ of linear image operation is a weighted sum of each pixel in the input $I$
		\begin{gather*}
			I_j'={\scriptstyle \sum_{i=1}^{N}}\alpha_{ij}I_i,\qquad j=1\cdots N
		\end{gather*}
	\item[\lp{Linear Filtering}] Linear operations can be written:
		\begin{gather*}
			I'(x,y)={\scriptstyle\sum_{i,j\in N(x,y)}^{}}K(x,y;i,j)I(i,j)
		\end{gather*}
		$I=\text{input image}$; $I'=\text{output of operation}$. $k$ is \emph{kernel} of the operation. $N(m,n)$ is a neighbourhood of $(m,n)$.
	\item[\lp{Correlation}] e.g. template matching. Linear operation: $I'=KI$ 
		\begin{gather*}
			I'(x,y)={\scriptstyle \sum_{i,j\in N(x,y)}^{}}K(i,j)I(x+i,y+j)
		\end{gather*}
	\item[\lp{Convolution}] e.g. point spread function
		\begin{align*}
			I'(x,y)={\scriptstyle\sum}_{i,j\in N(x,y)}^{} K(i,j)I(x-i,y-j)
		\end{align*}
	\item[\lp{Edge}] The filter window falls off the edge of the image, we need to extrapolate, methods:
		\begin{inparaenum}[\itshape(1)]
			\item clip filter (black)
			\item wrap around
			\item copy edge
			\item reflect across edge
			\item vary filter near edge
		\end{inparaenum}
	\item[\lp{Filter at boundary}]
		\begin{inparaenum}[\itshape(1)]
			\item ignore, copy or trucate. No processing of boundary pixels. Pad image with zeros (matlab). Pad image with copies of edge rows/columns
			\item truncate kernel
			\item reflected indexing
			\item circular indexing
		\end{inparaenum}
	\item[\lp{Separable Kernels}] Separable filters can be written 
		\begin{gather*}
			K(m,n)=f(m)g(n)
		\end{gather*}
		for a rectangular neighbourhood with size $(2M+1)\times (2N+1)$,
		\begin{align*}
			I'(m,n)&=f*(g*I(N(m,n))),\\
			I''(m,n)&={\scriptstyle\sum}_{j=-N}^{N}g(j)I(m,n-j),\\
			I'(m,n)&={\scriptstyle\sum}_{i=-M}^{M}f(i)I''(m-i,n).
		\end{align*}
		$\to (2M+1)+(2N+1)$ operations!
	\item[\lp{Smoothing kernels (low-pass filters)}] 
		\begin{inparaitem}
		\item Mean filter: $\displaystyle\frac{1}{9} \left(
				\begin{smallmatrix}
					1&1&1\\
					1&1&1\\
					1&1&1
				\end{smallmatrix} \right)$,
			\item Weighted smoothing filters: $\displaystyle\frac{1}{10} \left(
				\begin{smallmatrix}
					1&1&1\\
					1&2&1\\
					1&1&1
				\end{smallmatrix} \right)$,\,
			 $\displaystyle\frac{1}{16} \left(
				\begin{smallmatrix}
					1&2&1\\
					2&4&2\\
					1&2&1
				\end{smallmatrix} \right)$.
		\end{inparaitem}
	\item[\lp{Gaussian Kernel}]
	 Idea: Weight contributions of neighboring pixels:
		\begin{gather*}\scriptstyle
			\mathcal{N}_{\mu=0,\sigma}(x,y)=\frac{1}{2\pi\sigma^2}e^{-\frac{\left( x^2+y^2 \right)}{2\sigma^2}}
		\end{gather*}
		Smoothing with a Gaussian instead of a box filter removes the artefact of the vertical and horizontal lines. Gaussian smoothing Kernel is \emph{separable}! $\mathcal{N}(x,y)=\mathcal{N}(x)\mathcal{N}(y)$. Amount of smoothing depends on $\sigma$ and window size. $\text{Width}>3\sigma$.
	\item[\lp{Scale space}] Convolution of a Gaussian with $\sigma$ with itself is a gaussian with $\sigma\sqrt{2}$. Repeated convolution by a Gaussian filter produces the scale space of an image.
	\item[\lp{Gaussian filter top-5}] 
		\begin{inparaenum}[\itshape(1)]
			\item Rotationally symmetric.
			\item Has a single lobe. $\to$ Neighbor's influence decreases monotonically.
			\item Still one lobe in frequency domain. $\to$ No corruption from high frequencies
			\item Simple relationship to $\sigma$
			\item Easy to implement efficiently
		\end{inparaenum}
	\item[\lp{Differential filters}]
	\begin{inparaitem}
			\item Prewitt operator: $\displaystyle \left(
					\begin{smallmatrix}
						-1&0&1\\
						-1&0&1\\
						-1&0&1
					\end{smallmatrix} \right)$,
				\item Sobel operator: $\displaystyle \left(
					\begin{smallmatrix}
						-1&0&1\\
						-2&0&2\\
						-1&0&1
					\end{smallmatrix} \right)$.
			\end{inparaitem}
		\item[\lp{High-pass filters}] 
	\begin{inparaitem}
			\item Laplacian operator: $\displaystyle \left(
					\begin{smallmatrix}
						0&1&0\\
						1&-4&1\\
						0&1&0
					\end{smallmatrix} \right)$,
				\item High-pass filter: $\displaystyle \left(
					\begin{smallmatrix}
						-1&-1&-1\\
						-1&8&-1\\
						-1&-1&-1
					\end{smallmatrix} \right)$.
			\end{inparaitem}
		\item[\lp{Differentiation and convolution}]
			\begin{align*}
				&\pdv{f}{x}\\
				&\quad=\lim_{\varepsilon\to0}\left( \frac{f(x+\varepsilon,y)}{\varepsilon}-\frac{f(x,y)}{\varepsilon} \right)\\
				&\pdv{f}{x}
				&\quad\approx\frac{f\!\left( x_{n+1},y \right)-f\!\left( x_n,y \right)}{\Delta x},
			\end{align*}
			which is obviously a convolution $\left( \begin{smallmatrix}-1&1\end{smallmatrix} \right)$
		\item[\lp{Filters and templates}] Filters at some point can be sees as taking a $\cdot$-product between the image and some vector, the image is a set of dot products, filters look like the effects they are intended to find, filters find effects they look like.
		\item[\lp{Image sharpening}] Also known as enhancement. Increases the high frequency components to enhance edges. 
			\begin{gather*}
				I'=I+\alpha\abs{K*I},
			\end{gather*}
	 where $K$ is a high-pass filter kernel and $\alpha\in [0,1]$.
 \item[\lp{Integral images}] integral images (also known as summed-area tables) allow to efficiently compute the convolution with a constant rectangle
	 \begin{gather*}
		 \mathcal{I}(x,y)={\scriptstyle\int_{0}^{x}}\!\rd{x'}{\scriptstyle\int_{0}^{y}}\!\rd{y'}\, I(x',y')
	 \end{gather*}
	 \begin{align*}
		 A&=\mathcal{I}(1),\\
		 A+C&=\mathcal{I}(3),\\
		 A+B&=\mathcal{I}(2),\\
		 A+B+C+D&=\mathcal{I}(4).
	 \end{align*}
	 \begin{gather*}
		 D=\mathcal{I}(4)-\mathcal{I}(2)-\mathcal{I}(3)+\mathcal{I}(1)
	 \end{gather*}
	 Also possible along diagonal.
 \item[\lp{Viola-Jones cascade face detection}] Very efficient face detection using integral images.
\end{compactdesc}
