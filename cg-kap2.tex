\chapter{Light and Colors}
\begin{align*}
\text{Computer graphics}&=\text{generating images},\\
\text{Image}&=\text{array of pixels},
\end{align*}
each pixel represents one light ray (or more).
\begin{compactdesc}
	\item[\lp{What is light}] A form of electromagnetic (EM) radiation. Amplitude determies intensity. We perceive a limited section of the spectrum as ``visible light''
	\item[\lp{What is color}] Different wavelengths give different colors.
	\item[\lp{Spectral Distribution of Illumination}]
		Light can be a mixture of many wavelegths. Spectral power distribution (SPD) 
		\begin{gather*}
			P(\lambda)=\text{intensity at wavelegth }\lambda
		\end{gather*}
		We perceive these distributions as colors.
	\item[\lp{Measuring Light}] Each ray carries a spectrum $P(\lambda)$. $P(\lambda)$ contains more information than humans can and need to process. Humans ``project'' this sepctrum onto a lower-dimensional subspace.
	\item[\lp{Vector repetition}]
		\begin{align*}
			x&=x_1\vtr{n}_1+x_2\vtr{n}_2+x_3\vtr{n}_3\\
			x&=(\vtr{x}\cdot \vtr{n}_1)\vtr{n}_1+(\vtr{x}\cdot \vtr{n}_2)\vtr{n}_2\\
			&\quad+(\vtr{x}\cdot \vtr{n}_3)\vtr{n}_3
			\shortintertext{Projection onto 2D subspace}
			\vtr{x}^{\text{proj}}&=(\vtr{x}\cdot \vtr{n}_1)\vtr{n}_1+(\vtr{x}\cdot \vtr{n}_2)\vtr{n}_2
		\end{align*}
		An infinite dimensional vector is a function 
		\begin{align*}
			\vtr{x}^{\text{3D}}&=(x_1,x_2,x_3)\\
			\to\vtr{x}^{\text{inf}}&=x(\lambda)
			\shortintertext{Infinite number of basis functions needed. Projection onto 3D subspace with $\vtr{n}_1(\lambda),\vtr{n}_2(\lambda),\vtr{n}_3(\lambda)$ orthonormal basis functions}
			\vtr{x}^{\text{proj}}(\lambda)&=x_1\vtr{n}_1(\lambda)+x_2\vtr{n}_2(\lambda)\\
			&\quad+x_3\vtr{n}_3(\lambda)
			\shortintertext{Coordinates are continuous inner products}
			x_i&={\scriptstyle\int\limits_{}^{}}\dd{\lambda} \, x(\lambda)\vtr{n}_i(\lambda)
		\end{align*}
		\section{Human Color Perception}
		Humans project $P(\lambda)$ into a 3D subspace. Most mammals have 2 types of cones (2D subspace. (Mantis Shrimp use an 8D subspace.) We project infinite dimensional space onte 3D. Some information must be lost! Two completely different SPDs might look the same to us.
	\item[\lp{The CIE Primary System (1931)}] \emph{C}ommission \emph{I}nternationale de l'\emph{E}clairage. Setup for measuring human color sensitivity. Three light sources at: $\SI{435.8}{\nano\meter},\SI{546.1}{\nano\meter}$ and $\SI{700.0}{\nano\meter}$.
\section{Color Matching as Matrix Multiplication}
\begin{gather*}
	\!\left(\!\begin{smallmatrix}
		\text{Intensities for}\\
		\text{the three}\\
		\text{primary lights}
	\end{smallmatrix}\!\right)\!
	=
	\!\left(\!\begin{smallmatrix}
		\text{Color}\\
		\text{matching}\\
		\text{functions}
	\end{smallmatrix}\!\right)\!
	\!\left(\!\!\begin{smallmatrix}
		\text{SPD}\\
		\text{of test}\\
		\text{light}
	\end{smallmatrix}\!\!\right)\!\\
	\!\left(\!\begin{smallmatrix}
		R\\
		G\\
		B
	\end{smallmatrix}\!\right)\!
	=
	\!\left(\!\begin{smallmatrix}
		\overline{r}(\lambda_1)&\cdots&\overline{r}(\lambda_N)\\
		\overline{g}(\lambda_1)&\cdots&\overline{g}(\lambda_N)\\
		\overline{b}(\lambda_1)&\cdots&\overline{b}(\lambda_N)\\
	\end{smallmatrix}\!\right)\!
\end{gather*}
With $ { \!\left(\!\begin{smallmatrix} P(\lambda_1)&\cdots&P(\lambda_N) \end{smallmatrix}\!\right)\!}^T={\!\left(\!\begin{smallmatrix} 1&0&\cdots&0 \end{smallmatrix}\!\right)\!}^T$ for monochromatic test light.
\end{compactdesc}
