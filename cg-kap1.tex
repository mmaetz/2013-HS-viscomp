\lstset{ %
	language=c++
}
\makepart{Computer Graphics}
\setcounter{chapter}{0}
\chapter{The Graphics Pipeline}
From geometry and lighting to pixels\\
Inputs:\\
\begin{enumerate*}[label=\protect\circled{\arabic*},itemjoin=]
	\item Geometry\\
		\begin{itemize*}[label=\colorbullet,itemjoin=]
			\item Triangles,points,lines,other primitives\\
		\end{itemize*}
	\item Materias and Lighting\\
	\item Virtual Camera\\
		\begin{itemize*}[label=\colorbullet,itemjoin=]
			\item Camera position,Frustum\\
		\end{itemize*}
\end{enumerate*}
$\to$ The Graphics Pipeline \\$\to$ Outputs: Colors for Display.
\section{Scheme}
Modeling Transform $\to$ Viewing Transform $\to$ 3D Clipping $\to$ Lightning, Shading, Texturing $\to$ Projection to Screen Space $\to$ Scan Conversion.
\section{Transform}
\begin{compactdesc}
	\item[\lp{Modeling Transform}] FRom object to world space
	\item[\lp{Viewing Transform}] From world to camera space
	\item[\lp{3D Clipping}] Remove parts of objects outside the frustum. Needed to avoid unnecessary computations and numerical instabilities.
	\item[\lp{Lighting, Shading, Texturing}] Compute color based on lighting, materials, surface normals.
	\item[\lp{Projection to Screen Space}] Project from 3D to 2 screen space
	\item[\lp{Scan Conversion}] \hfill\\
		\begin{enumerate*}[label=\protect\circled{\arabic*},itemjoin=]
			\item Compute frament attibutes\\
			\item  Interpolate per-vertex values\\
			\item Discretize continuous primitives\\
			\item Triangles, lines, polygons
		\end{enumerate*}
\section{Programmer's View} 
\item[\lp{Contemporary pipeline}]\hfill\\
\begin{enumerate*}[label=\protect\circled{\arabic*},itemjoin=]
	\item CPU\\ 
	\item Vertex Processing $\leftarrow$ programmable\\
		\begin{enumerate*}[label=\quad\protect\circled{\alph*},itemjoin=]
			\item Per-vertex operations\\
			\item Transforms \& Lighting\\
			\item Flow control\\
		\end{enumerate*}
	\item Rasterization\\
	\item Fragment Processing $\leftarrow$ programmable\\
		\begin{enumerate*}[label=\quad\protect\circled{\alph*},itemjoin=]
			\item Per-fragment operations\\
			\item Shading \& Texturing\\
			\item Blending\\
		\end{enumerate*}
	\item Display\\
\end{enumerate*}
\item[\lp{Programming with ``shaders''}] \hfill\\
			Attributes (per-vertex) $\to$\\
	\begin{enumerate*}[label=\protect\circled{\arabic*},itemjoin=]
		\item Vertex Shader\\
			$\to$ Varying (per-vertex)$\to$\\
		\item Interpolation\\
			$\to$ Varying (per-fragment)$\to$\\
		\item Fragment Shader\\
			$\to$ Fragment Color (per-fragment)\\
	\end{enumerate*}
	With the Vertex Shader and Fragment Shader being Uniforms (constants). Attributes given per vertex. Vertex Shader computes varying. Interpolate varying values. Fragment Shader computes pixel color.
\section{Vertex Shader}
\begin{lstlisting}
varying vec3 lightDirection, normal;
void main()
{
	vec4 vertexPosition=gl_ModelViewMatrix*g_Vertex;
	lightDirection=vec3(g_LightSource[0].position-vertexPosition);
	normal=gl_Normal;
	gl_Position=vertexPosition;
}
\end{lstlisting}
where on line 1 are the variables to be interpolated and passed to the fragment shader. In the body of main() the per-vertex attributes are computed.
\section{Fragment Shader}
\begin{lstlisting}
varying vec3 lightDirection, normal;

void main()
{
	vec3 lightDirectionNormalized=normalize(lightDirection);
	float intensity=dot(lightDirectionNormalized,normal);
	gl_FragColor=vec4(intensity,intensity,intensity,1.0);
}
\end{lstlisting}
where in line 1 are the interpolated variables passed from vertex shader. In the main() body the color of this fragment (gl-FragColor) is computed.
\item[\lp{Fragment}] $\text{Fragment}\neq\text{Pixel}$. A Fragment can store: Color, Position, Depth, Texture Coordinates, Window ID,\ldots\\
	\section{Graphics APIs}
	\emph{A}pplication \emph{P}rogramming \emph{I}nterfaces. Give access to the graphics hardware. Are hardware-independent and abstract away complex details.
	\section{OpenGL}
\item[\lp{includes}]\hfill\\
	\begin{lstlisting}
	#include <GLee.h>
	#include <GL/gl.h>
	#include <GL,glut.h>
	\end{lstlisting}
\item[\lp{main application}]\hfill\\
	\begin{lstlisting}
	int main(int argc, char **argv)
	{
		// Initialize the glut environment
		glutInit(&argc,argv);
		glutInitDisplayMode(GLUT_DOUBLE | GLUT_RGB | GLUT_DEPTH)

		// Create a new window named Simple
		glutCreateWindow("Simple");

		// Register the display callback function
		glutDisplayFunc(displayFunc);

		// Hand over the control to GLUT server loop
		glutMainLoop();

		return 0;
	}
	\end{lstlisting}
\item[\lp{display function}]\hfill\\
	\begin{lstlisting}
	void displayFunc(void)
	{
		glClear(GL_COLOR_BUFFER_BIT | GL_DEPTH_BUFFER_BIT);
		GLuint vshader=glCreateSHader(GL_VERTEX_SHADER);
		glCompileShader(vshader);

		GLuint fshader = glCreateShader(GL_FRAGMENT_SHADER);
		glShaderSource(fshader,1,&fragment_p,NULL);
		glCompileShader(fshader);
		GLuint shaderprogram=glCreateProgram();
		glAttachShader(shaderprogram, vshader);
		glAttachShader(shaderprogram, fshader);
		glLinkProgram(shaderprogram);

		glUseProgram(shaderprogram);

		float coords[] = {-1, -1, 0, 1, -1, 0, 0, 1, 0};
		GLint location=glGetAttribLocation(shaderprogram,"pos");
		glEnableVertexAttribArray(location);
		gVertexAttribPointer(location, 3, GL_FLOAT, GL_FALSE, 3*sizeof(float), coords);
		glDrawArrays(GL_TRIANGLES, 0, 3);

		glutSwapBuffers();
	}
	\end{lstlisting}
\item[\lp{vertex shader}] \hfill\\
	\begin{lstlisting}
	attribute vec4 pos;
	varying vec4 color_out;

	void main()
	{
		if(pos.x == -1.0) color_out = vec(1,0,0,1);
		if(pos.x == 1.0) color_out = vec(0,1,0,1);
		if(pos.x == 0.0) color_out = vec(0,0,1,1);

		gl_Position = pos;
	}
	\end{lstlisting}
\item[\lp{fragment shader}] \hfill\\
	\begin{lstlisting}
	varying vec4 color_out;

	void main()
	{
		gl_FragColor = color_out;
	}
	\end{lstlisting}
\item[\lp{Integration}]
	\begin{itemize*}[label=\colorbullet]
		\item gl.lib: main library\\
		\item glu.lib: utilities. Supports NURBS, complex polygons, quadric shapes, etc.\\
		\item glut.lib: toolkit. Simplifies and abstracts window handling
	\end{itemize*}
\item[\lp{Primitives}]\hfill\\
	\begin{lstlisting}
	glDRawElements(X, count, GL_UNSIGNED_INT, 0);
	\end{lstlisting}
	X:\hfill\\
	\begin{enumerate*}[label=\protect\circled{\arabic*},itemjoin=]
		\item GL\_POINTS,\\
		\item GL\_LINES,\\
		\item GL\_POLYGON,\\
		\item GL\_TRIANGLES,\\
		\item GL\_QUADS,\\
		\item GL\_TRIANGLE\_STRIP,\\
		\item GL\_TRANGLE\_FAN,\\
	\end{enumerate*}
\item[\lp{Extensions}] \hfill\\
	\begin{enumerate*}[label=\protect\circled{\arabic*},itemjoin=]
		\item Access to hardware-specific features\\
		\item Specified by 3D video card manufacturers and OpenGL vendors\\
		\item Defined in \texttt{glext.h}\\
		\item Extensions have manufacturer's postfix.
	\end{enumerate*}
\end{compactdesc}
