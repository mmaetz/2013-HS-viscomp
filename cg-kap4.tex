\chapter{Lightning and Shading}
\section{Measuring Light}
\begin{compactdesc}
	\item[\lp{Measuring}] $\to$ counting photons
	\item[\lp{Radiometry}] Studies the measurement of electromagnetic radiation, include visible light
\section{Basic Definitions}
Angle:
\begin{align*}
	\theta&=\frac{\ell}{r},&& \text{circle $2\pi$ radians}
	\shortintertext{Solid angle}
	\Omega&=\frac{A}{r^2},
	&&\text{sphere: $4\pi$ steradians}
\end{align*}
\item[\lp{Direction}]
	point on the unit sphere, parametrized by two angles $\vtr{\omega}(\theta,\phi)$
	\begin{align*}
		\omega_{\vtr{x}}&=\sin\theta\cos\phi\\
		\omega_{\vtr{y}}&=\sin\theta\sin\phi\\
		\omega_{\vtr{z}}&=\cos\theta
	\end{align*}
\item[\lp{Differential Solid Angle}]
	\begin{align*}
		\dd{A}&=(r\dd{\theta})\left( r\sin\theta\dd{\phi} \right)\\
		\dd{\vtr{\omega}}&=\frac{\dd{A}}{r^2}=\sin\theta\dd{\theta}\dd{\phi}\\
		\Omega&={\scriptstyle\int_{S^2}^{}}\dd{\vtr{\omega}}\\
		&={\scriptstyle\int_{0}^{2\pi}}\dd{\phi}{\scriptstyle\int_{0}^{\pi}}\dd{\theta}\, \sin\theta=4\pi
	\end{align*}
	Assume light consists of photons with 
	\begin{align*}
		\vtr{x}:&\quad\text{Position},\\
		\vtr{\omega}:&\quad\text{Direction of motion},\\
		\lambda:&\quad\text{Wavelength}.
	\end{align*}
	Each photon has an energy of:
	\begin{gather*}
		\frac{hc}{\lambda}
	\end{gather*}
	\begin{align*}
		\shortintertext{Plack's constant}
		h&\approx \SI{6.63e-34}{\meter\squared\kilogram\per\second}\\
		\shortintertext{speed of light in vacuum}
		c&=\SI{299790458}{\meter\per\second}
		\shortintertext{Unitof energy}
		J&=\si{\kilogram\meter\squared\per\second\squared}
	\end{align*}
	\section{Radiometry}
		\item[\lp{flux}]
			total amonut of energy passing through surface or space per unit time
			\begin{align*}
				\Phi(A)&&\left[ \frac{J}{s} \right]=[ W ]
			\end{align*}
			examples: Number of photons hitting a wall per secound, number of photons leaving a lightbulb per second.
			\begin{gather*}
				E(\vtr{x})=\dv{\Phi(A)}{A(\vtr{x})}\\
				{\scriptstyle\int_{A}^{}}\dd{A(\vtr{x})}\,E(\vtr{x})=\Phi(A)\\
				{\scriptstyle\int\limits_{A}^{}}\dd{A}(\vtr{x}){\scriptstyle\int\limits_{H^2}^{}}\dd{\vtr{\omega}}\, L(\vtr{x},\vtr{\omega})\cos\theta=\Phi(A)
			\end{gather*}
		\item[\lp{irradiance}]
			Flux per unit area \emph{arriving} at a surface $\to$ area density of flux
			\begin{align*}
				E(\vtr{x})=\dv{\Phi(A)}{A(\vtr{x})} &&\left[ \frac{W}{m^2} \right].
			\end{align*}
			Example: Number of photons hitting a small patch of a wall per second, diivded by the size of the patch.
	\pgr{3-1_LightingAndShadingI}{12}{-2.4}{-0.5}{-0.5}{0.15}
	\begin{gather*}
		E=\frac{\Phi}{A}
	\end{gather*}
	\pgr{3-1_LightingAndShadingI}{12}{0.2}{-0.8}{2.2}{0.15}
	Lambert's cosine law
	\begin{gather*}
		E=\frac{\Phi}{A/\cos\theta}=\frac{\Phi}{A}\cos\theta
	\end{gather*}
	\begin{align*}
		\mathcal{L}(\vtr{x},\vtr{\omega})&=\dvt{\Phi(A)}{\cos\theta\dd{A}(\vtr{x})\dd{\vtr{\omega}}}\\\\
		\leadsto\mathcal{L}(\vtr{x},\vtr{\omega})&=\dv{E(\vtr{x})}{\cos\theta \dd{\vtr{\omega}}}
	\end{align*}
	\begin{gather*}
		\leadsto\mathcal{L}(\vtr{x},\vtr{\omega})\cos\theta\dd{\vtr{\omega}}=\dd{E(\vtr{x})}
	\end{gather*}
	Integrate radiance over the hemisphere. Same for radiosity.
		\item[\lp{radiosity}]
			Flux per unit area \emph{leaving} at a surface $\to$ area density of flux
			\begin{align*}
				B(\vtr{x})=\dv{\Phi(A)}{A(\vtr{x})} &&\left[ \frac{\si{\watt}}{\si{\meter\squared}} \right].
			\end{align*}
			Example: Number of photons reflecting off a small patch of a wall per second, divided by the size of the patch. 		\item[\lp{Radiant intensity}]
			Power (flux) per solid angle$=$directional density of flux
			\begin{align*}
				I(\vtr{\omega})&=\dv{\Phi}{\vtr{\omega}}&& \left[ \frac{\si{\watt}}{\si{\steradian}} \right]\\
				\Phi&={\scriptstyle\int_{S^2}^{}}\dd{\vtr{\omega}}\, I(\vtr{\omega})
			\end{align*}
			Example: Power per unit solid angle emanating from a point source.
			\begin{gather*}
				\Phi=4\pi I,
			\end{gather*}
			isotropic point source.
		\item[\lp{Radiance}]
			Intensity per unit area$=$flux density per unit solid angle, per perpendicular unit area
			\begin{align*}
				\mathcal{L}(\vtr{x},\vtr{\omega})&=\dv{I(\vtr{\omega})}{A(\vtr{x})}
				=\pdvv{\Phi}{\vtr{\omega}}{A^{\perp}(\vtr{x},\vtr{\omega})}\\
				&=\pdvv{\Phi(A)}{\vtr{\omega}}{A(\vtr{x})\cos\theta}\quad \left[ \frac{\si{\watt}}{\si{\meter\squared\steradian}} \right]
			\end{align*}
			Most fundamental for raytracing. Remains constant along a ray.
\section{Reflection Models}
\item[\lp{BRDF}] \emph{B}idirectional \emph{R}eflectance \emph{D}istribution \emph{F}unction.
	\begin{align*}
		f_r(\vtr{x},\omega_i,\omega_r)&=\dv{L_r(\vtr{x},\vtr{\omega}_r)}{E_i(\vtr{x},\vtr{\omega}_i)}\\
		&=\frac{\dd{L_r(\vtr{x},\vtr{\omega}_r)}}{L_i(\vtr{x},\vtr{\omega}_i)\cos\theta_i\dd{\vtr{\omega}}_i}
	\end{align*}
	Differential irradiance due to a cone of directions around $\vtr{\omega}_i$.
\end{compactdesc}
